% This is samplepaper.tex, a sample chapter demonstrating the
% LLNCS macro package for Springer Computer Science proceedings;
% Version 2.21 of 2022/01/12
%
\documentclass[runningheads]{llncs}
%
\usepackage[T1]{fontenc}
% T1 fonts will be used to generate the final print and online PDFs,
% so please use T1 fonts in your manuscript whenever possible.
% Other font encondings may result in incorrect characters.
%
\usepackage{graphicx}
% Used for displaying a sample figure. If possible, figure files should
% be included in EPS format.
%
% If you use the hyperref package, please uncomment the following two lines
% to display URLs in blue roman font according to Springer's eBook style:
%\usepackage{color}
%\renewcommand\UrlFont{\color{blue}\rmfamily}
%\urlstyle{rm}
%
\begin{document}
%
\title{Schema Mechanisms as an Attempt to Implement Genetic Epistemology}
%
\titlerunning{Schema Mechanisms}
% If the paper title is too long for the running head, you can set
% an abbreviated paper title here
%
\author{Olivier L. Georgeon\inst{1}\orcidID{0000-0003-4883-8702} \and
Filipo Perotto\inst{2}\orcidID{1111-2222-3333-4444} \and
Third Author\inst{3}\orcidID{2222--3333-4444-5555}}
%
\authorrunning{O. Georgeon \& F. Perotto}
% First names are abbreviated in the running head.
% If there are more than two authors, 'et al.' is used.
%
\institute{UR CONFLUENCE: Sciences et Humanites (EA 1598), UCLy, France \email{ogeorgeon@univ-catholyon.fr}\\
	\and ONERA, France \email{filipo.perotto@onera.fr}
 \and
 }
%
\maketitle              % typeset the header of the contribution
%
\begin{abstract}
We review schema mechanisms.

\keywords{Schema mechanism  \and Genetic epistemology \and Constructivist learning.}
\end{abstract}
%
%
%
\section{Introduction}

\section{Genetic epistemology}

The notion of \textit{sensori-motor scheme} proposed by Piaget. 

Piaget's genetic epistemology: 
``Knowledge does not originally arise either from a subject conscious of itself or from objects already constituted (from the subject's point of view) that would impose themselves on the subject. 
Knowledge results from interactions occurring halfway between the subject and the objects, and thus involving both, but due to a complete un-differentiation and not from exchanges between distinct forms.

If, at the beginning, there is neither a subject, in the epistemic sense of the term, nor objects, conceived as such, nor, above all, invariant instruments of exchange, then the initial problem of knowledge will be to construct such mediators. 
Starting from the contact zone between one's own body and the objects, these mediators will progressively engage more deeply in both complementary directions toward the exterior and the interior. 
It is from this dual progressive construction that the joint elaboration of both the subject and the objects depends.

The initial instrument of exchange is not perception, as rationalists too easily conceded to empiricism, but rather action itself, with its much greater plasticity. 
Certainly, perceptions play an essential role, but they partly depend on action as a whole, and some perceptual mechanisms that one might have thought to be innate or very primitive only emerge at a certain level of object construction.'' (\cite{piaget_principles_1997}
)


\cite{ziemke_construction_2001}.

\section{Schema mechanisms}

 
\cite{drescher_made-up_1991}
\cite{chaput_constructivist_2004}
\cite{georgeon_intrinsically-motivated_2012} 
\cite{perotto_computational_nodate} 
\cite{guerin_piagetian_2008}
\cite{wang_new_2012}
\section{Conclusion}

The problem of abstraction. 


\begin{credits}
\subsubsection{\ackname} .

\subsubsection{\discintname}
The authors have no competing interests to declare that are
relevant to the content of this article.
\end{credits}
%
% ---- Bibliography ----
%
% BibTeX users should specify bibliography style 'splncs04'.
% References will then be sorted and formatted in the correct style.
%
\bibliographystyle{splncs04}
\bibliography{ConstructivistAI.bib}
%
\end{document}
